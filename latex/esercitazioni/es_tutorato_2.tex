\documentclass[11pt,a4]{article}

\usepackage[margin=2cm]{geometry}

\usepackage{multicol}

\usepackage{amsmath}
\usepackage{url}

\usepackage{amsmath}
\usepackage{url}

\usepackage[utf8]{inputenc}

% Default fixed font does not support bold face
\DeclareFixedFont{\ttb}{T1}{txtt}{bx}{n}{10} % for bold
\DeclareFixedFont{\ttm}{T1}{txtt}{m}{n}{10}  % for normal

% Custom colors
\usepackage{color}
\definecolor{deepblue}{rgb}{0,0,0.5}
\definecolor{deepred}{rgb}{0.6,0,0}
\definecolor{deepgreen}{rgb}{0,0.5,0}

\usepackage{listings}

% Python style for highlighting
\newcommand\pythonstyle{\lstset{
language=Python,
basicstyle=\ttm,
otherkeywords={self},             % Add keywords here
keywordstyle=\ttb\color{deepblue},
emph={MyClass,__init__},          % Custom highlighting
emphstyle=\ttb\color{deepred},    % Custom highlighting style
stringstyle=\color{deepgreen},
frame=tb,                         % Any extra options here
showstringspaces=false            % 
}}


% Python environment
\lstnewenvironment{python}[1][]
{
\pythonstyle
\lstset{#1}
}
{}

% Python for external files
\newcommand\pythonexternal[2][]{{
\pythonstyle
\lstinputlisting[#1]{#2}}}

% Python for inline
\newcommand\pythoninline[1]{{\pythonstyle\lstinline!#1!}}


\usepackage{collectbox}

\newcommand{\mybox}[2]{$\quad$\fbox{
\begin{minipage}{#1cm}
\hfill\vspace{#2cm}
\end{minipage}
}}


\usepackage{fancyhdr}
\pagestyle{fancy}
\rhead{Programmazione 1 - Esercitazione 5}

\usepackage[T1]{fontenc}
\usepackage[utf8]{inputenc}
\usepackage{lmodern}
%%%%%%%%%%%%%%%%%%%%%%%%%%%%%%%%%%%%%%%%%%%%%%%%%%%%%%%%%
% Source: http://en.wikibooks.org/wiki/LaTeX/Hyperlinks %
%%%%%%%%%%%%%%%%%%%%%%%%%%%%%%%%%%%%%%%%%%%%%%%%%%%%%%%%%
\usepackage{hyperref}
\usepackage{graphicx}
\usepackage[english]{babel}

\usepackage{bm}
\usepackage{amsmath}
\usepackage{amsfonts}

\usepackage{amsthm}
\newtheorem{definition}{Definizione}
\newtheorem{theorem}{Teorema}
\renewcommand*{\proofname}{Dimostrazione}
\newtheorem{example}{Esempio}
\newtheorem{lemma}{Lemma}
\newtheorem{exercise}{Esercizio}
\newtheorem{property}{Proprietà}

\usepackage[ruled,vlined,linesnumbered]{algorithm2e}

\newcommand{\xstar}{x^*}
\newcommand{\bxstar}{\bm{x^*}}
\newcommand{\bx}{\bm{x}}
\newcommand{\Rn}{\mathbb{R}^n}
\newcommand{\RR}{\mathbb{R}}
\newcommand{\norm}[1]{\left\lvert \left\lvert #1 \right\lvert \right\lvert}

\newcommand{\fx}{f(x)}

\newcommand{\gradfx}{\nabla \fx}
\newcommand{\Gx}{\nabla f(x)}
\newcommand{\Gk}{\nabla f(x_k)}
\newcommand{\Gs}{\nabla f(\xstar)}

\newcommand{\Hx}{\nabla^2 f(x)}
\newcommand{\Hk}{\nabla^2 f(x_k)}
\newcommand{\Hs}{\nabla^2 f(\xstar)}
\newcommand{\hess}{\nabla^2 f}

\newcommand{\step}{\alpha}
\newcommand{\Seqx}{\{ x_k \}}

\usepackage{mathtools}
\newcommand\myeq{\stackrel{\mathclap{\normalfont\mbox{def}}}{=}}

\usepackage{listings}
\lstset
{ 
    language=Matlab,
    basicstyle=\normalsize,
    numbers=left,
    stepnumber=1,
    showstringspaces=false,
    tabsize=1,
    breaklines=true,
    breakatwhitespace=false,
   frame=single
}


\begin{document}
\thispagestyle{empty}
\hrule
\begin{center}
   {\Large {\bf Programmazione 1 \hspace{3cm} $\quad \quad \quad$ Tutorato 3}}
\end{center}

\hrule

%%%%%%%%%%%%%%%%%%%%%%%%%%%%%%%%%%%%%%%%%%%%%%%%%%%%%%%%%%%%%%%%%%%%%%%%%%%%%
\section*{}

\begin{enumerate}

%%%%%%%%%%%%%%%%%%%%%%%%%%%%%%%%%%%%%%%%%%%%%%%%%%%%%%%%%%%%%%%%%%%%%%%%%%%%%
\item Scrivere due funzioni {\tt MapFR(x)} e {\tt FilterFR(x)} che implementano la $map$ e la $filter$ utilizzando la FoldRight.

\begin{python}
def FoldRight(P, As, v):
    if IsEmpty(As):
        return v
    return P(Head(As), FoldRight(P, Tail(As), v))
\end{python}


%%%%%%%%%%%%%%%%%%%%%%%%%%%%%%%%%%%%%%%%%%%%%%%%%%%%%%%%%%%%%%%%%%%%%%%%%%%%%
\item Scrivere due funzioni {\tt MapFL(x)} e {\tt FilterFL(x)} che implementano la $map$ e la $filter$ utilizzando la FoldLeft.

\begin{python}
def FoldLeft(P, As, v):
    if IsEmpty(As):
        return v
    return FoldLeft(P,Tail(As),P(z,Head(As))
\end{python}

%%%%%%%%%%%%%%%%%%%%%%%%%%%%%%%%%%%%%%%%%%%%%%%%%%%%%%%%%%%%%%%%%%%%%%%%%%%%%
\item Ricordando le strutture viste:

\begin{multicols}{2}
\small
\fbox{\parbox[t][3.5cm][t]{7.4cm}{
\vspace{0.2cm}
\textbf{def} $F(X_1, \dots, X_n):$\\
$\hspace*{0.5cm}$ \textbf{def} $Fiter(Y_1,...,Y_m):$\\
$\hspace*{1.0cm}$    \textbf{if} $Proposizione(Y_1,...,Y_m)==True:$\\
\vspace{0.2cm}
$\hspace*{1.5cm}$        \textbf{return} $Y_1,...,Y_{m'}$ $(m'\le m)$\\
\vspace{0.2cm}
$\hspace*{1.0cm}$    \textbf{return}  $Fiter(\tilde{Y_1},...,\tilde{Y_m})$\\
$\hspace*{0.5cm}$    $\textbf{return}$ $Fiter(\tilde{X_1},...,\tilde{X_m})$\\}}
\columnbreak

\fbox{\parbox[t][3.5cm][t]{7.4cm}{
\vspace{0.2cm}
\textbf{def} $F(X_1, \dots, X_n):$\\
$\hspace*{0.5cm}$    $Y_i=\tilde{X_i}$ per $i=0,\dots,m$\\
$\hspace*{0.5cm}$ (le $\tilde{X_i}$ sono costruite a partire dalle $X_i$)\\ 
$\hspace*{0.5cm}$    \textbf{while} $Proposizione(Y_1,...,Y_m)==False:$\\
$\hspace*{1.4cm}$     $Y_i=\tilde{Y_i}$ per $i=0,\dots,m$\\
\\
$\hspace*{0.5cm}$    \textbf{return}  $Y_1,...,Y_{m'}$}}

\end{multicols}


Si scrivano in entrambi i modi le funzioni seguenti:

\begin{enumerate}
\item {\tt Fib(n)}:  prende in input un numero intero $n$ e restituisce l'n-esimo elemento della successione di Fibonacci.
\item {\tt Fn(n)}: prende in input un numero intero $n$ e restituisce l'n-esimo elemento della successione $Fn$ definita ricorsivamente:
\begin{equation*}
\mbox{Fn}(x) = \left\{ \begin{array}{ll}
 n & \mbox{ se } n<3, \\
 f(n-1)f(n-2)+nf(n-3) & \mbox{ se } n\ge3.
\end{array} \right.
\end{equation*}
\item {\tt Equal(As,Bs)}: prende in input due liste e restituisce {True} se sono uguali, {False} altrimenti.
\item {\tt PrimiPrimi(n)}: prende in input un numero intero $n$ e restituisce una lista dei primi n numeri primi.
\end{enumerate}
Dando una stima in termini di O-grande della complessità dell'algoritmo.

\end{enumerate}

\end{document}