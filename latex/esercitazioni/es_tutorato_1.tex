\documentclass[11pt,a4]{article}

\usepackage[margin=2cm]{geometry}

\usepackage{amsmath}
\usepackage{url}

\usepackage{amsmath}
\usepackage{url}

\usepackage[utf8]{inputenc}

% Default fixed font does not support bold face
\DeclareFixedFont{\ttb}{T1}{txtt}{bx}{n}{10} % for bold
\DeclareFixedFont{\ttm}{T1}{txtt}{m}{n}{10}  % for normal

% Custom colors
\usepackage{color}
\definecolor{deepblue}{rgb}{0,0,0.5}
\definecolor{deepred}{rgb}{0.6,0,0}
\definecolor{deepgreen}{rgb}{0,0.5,0}

\usepackage{listings}

% Python style for highlighting
\newcommand\pythonstyle{\lstset{
language=Python,
basicstyle=\ttm,
otherkeywords={self},             % Add keywords here
keywordstyle=\ttb\color{deepblue},
emph={MyClass,__init__},          % Custom highlighting
emphstyle=\ttb\color{deepred},    % Custom highlighting style
stringstyle=\color{deepgreen},
frame=tb,                         % Any extra options here
showstringspaces=false            % 
}}


% Python environment
\lstnewenvironment{python}[1][]
{
\pythonstyle
\lstset{#1}
}
{}

% Python for external files
\newcommand\pythonexternal[2][]{{
\pythonstyle
\lstinputlisting[#1]{#2}}}

% Python for inline
\newcommand\pythoninline[1]{{\pythonstyle\lstinline!#1!}}


\usepackage{collectbox}

\newcommand{\mybox}[2]{$\quad$\fbox{
\begin{minipage}{#1cm}
\hfill\vspace{#2cm}
\end{minipage}
}}


\usepackage{fancyhdr}
\pagestyle{fancy}
\rhead{Programmazione 1 - Esercitazione 5}

\usepackage[T1]{fontenc}
\usepackage[utf8]{inputenc}
\usepackage{lmodern}
%%%%%%%%%%%%%%%%%%%%%%%%%%%%%%%%%%%%%%%%%%%%%%%%%%%%%%%%%
% Source: http://en.wikibooks.org/wiki/LaTeX/Hyperlinks %
%%%%%%%%%%%%%%%%%%%%%%%%%%%%%%%%%%%%%%%%%%%%%%%%%%%%%%%%%
\usepackage{hyperref}
\usepackage{graphicx}
\usepackage[english]{babel}

\usepackage{bm}
\usepackage{amsmath}
\usepackage{amsfonts}

\usepackage{amsthm}
\newtheorem{definition}{Definizione}
\newtheorem{theorem}{Teorema}
\renewcommand*{\proofname}{Dimostrazione}
\newtheorem{example}{Esempio}
\newtheorem{lemma}{Lemma}
\newtheorem{exercise}{Esercizio}
\newtheorem{property}{Proprietà}

\usepackage[ruled,vlined,linesnumbered]{algorithm2e}

\newcommand{\xstar}{x^*}
\newcommand{\bxstar}{\bm{x^*}}
\newcommand{\bx}{\bm{x}}
\newcommand{\Rn}{\mathbb{R}^n}
\newcommand{\RR}{\mathbb{R}}
\newcommand{\norm}[1]{\left\lvert \left\lvert #1 \right\lvert \right\lvert}

\newcommand{\fx}{f(x)}

\newcommand{\gradfx}{\nabla \fx}
\newcommand{\Gx}{\nabla f(x)}
\newcommand{\Gk}{\nabla f(x_k)}
\newcommand{\Gs}{\nabla f(\xstar)}

\newcommand{\Hx}{\nabla^2 f(x)}
\newcommand{\Hk}{\nabla^2 f(x_k)}
\newcommand{\Hs}{\nabla^2 f(\xstar)}
\newcommand{\hess}{\nabla^2 f}

\newcommand{\step}{\alpha}
\newcommand{\Seqx}{\{ x_k \}}

\usepackage{mathtools}
\newcommand\myeq{\stackrel{\mathclap{\normalfont\mbox{def}}}{=}}

\usepackage{listings}
\lstset
{ 
    language=Matlab,
    basicstyle=\normalsize,
    numbers=left,
    stepnumber=1,
    showstringspaces=false,
    tabsize=1,
    breaklines=true,
    breakatwhitespace=false,
   frame=single
}


\begin{document}
\thispagestyle{empty}
\hrule
\begin{center}
   {\Large {\bf Programmazione 1 \hspace{3cm} $\quad \quad \quad$ Tutorato 1}}
\end{center}

\hrule

%%%%%%%%%%%%%%%%%%%%%%%%%%%%%%%%%%%%%%%%%%%%%%%%%%%%%%%%%%%%%%%%%%%%%%%%%%%%%
\section*{}

\begin{enumerate}

%%%%%%%%%%%%%%%%%%%%%%%%%%%%%%%%%%%%%%%%%%%%%%%%%%%%%%%%%%%%%%%%%%%%%%%%%%%%%
\item Scrivere una funzione {\tt Sign(x)} che prende in input un numero $x$ e ne restituisce il segno, ovvero
restituisce:
\begin{equation*}
\mbox{sign}(x) = \left\{ \begin{array}{ll}
 -1 & \mbox{ se } x<0, \\
 0 &  \mbox{ se } x = 0, \\
 1 & \mbox{ se } x>0.
\end{array} \right.
\end{equation*}

%%%%%%%%%%%%%%%%%%%%%%%%%%%%%%%%%%%%%%%%%%%%%%%%%%%%%%%%%%%%%%%%%%%%%%%%%%%%%
\item Scrivere una funzione {\tt Compose(f, g)} che prende in input due funzioni $f,g : \mathbb{R} \rightarrow \mathbb{R}$
e restituisce in output una funzione che calcola i due valori: $a=f(g(x))$ e $b=g(f(x))$.
Per esempio: 

\begin{python}
h = Compose(lambda x: x**2, lambda x: 7/x)
print(h(2))
#Out[]: (12.25, 1.75)
\end{python}

%%%%%%%%%%%%%%%%%%%%%%%%%%%%%%%%%%%%%%%%%%%%%%%%%%%%%%%%%%%%%%%%%%%%%%%%%%%%%
\item Usando le {\tt pairslist}, scrivere una funzione {\tt Halve(As)} che prende in input una lista
{\tt As} e restituisce in output due liste, la prima contenente la prima metà di elementi di As,
la seconda lista la seconda metà di elementi. 
Per esempio: {\tt Halve((1,(2,(3,(4,(13,(21,None)))))))} deve restituire le due liste
{\tt (1,(2,(3,None))), (4,(13,(21,None)))}.


%%%%%%%%%%%%%%%%%%%%%%%%%%%%%%%%%%%%%%%%%%%%%%%%%%%%%%%%%%%%%%%%%%%%%%%%%%%%%
\item Usando le {\tt pairslist}, scrivere una funzione {\tt Zip(As, Bs)} che prende in input due liste
{\tt As} e {\tt Bs} e restituisce in output una lista di coppie, contenente le coppie degli $i$-esimi 
elementi delle due liste. Per esempio: {\tt Zip((1,(2,(3,None))), (3,(4,(5,None))))} deve restituire la lista
{\tt ((1,3), ((2,4), ((3,6), None)))}.


%%%%%%%%%%%%%%%%%%%%%%%%%%%%%%%%%%%%%%%%%%%%%%%%%%%%%%%%%%%%%%%%%%%%%%%%%%%%%
\item Usando le {\tt pairslist}, scrivere una funzione {\tt Pairs(As)} che prende in input la lista
{\tt As} e restituisce in output una lista di coppie, contenente le coppie degli $(i, i+1)$ per $i =1,\dots,n-1$, dove
$n$ è la lunghezza della lista. Per esempio: {\tt Pairs((1,(2,(3,None))))} deve restituire la lista
{\tt ((1,2), ((2,3), None))}.


%%%%%%%%%%%%%%%%%%%%%%%%%%%%%%%%%%%%%%%%%%%%%%%%%%%%%%%%%%%%%%%%%%%%%%%%%%%%%
\item Usando le {\tt pairslist}, scrivere un predicato {\tt IsSorted(As)} che 
restituisce {\tt True} se la lista di numeri interi {\tt As} è ordinata in modo decrescente, e {\tt False}
altrimenti.

%%%%%%%%%%%%%%%%%%%%%%%%%%%%%%%%%%%%%%%%%%%%%%%%%%%%%%%%%%%%%%%%%%%%%%%%%%%%%
\item Si consideri la funzione {\tt FoldRight(P, As, v)} vista a lezione:

\begin{python}
def FoldRight(P, As, v):
    if IsEmpty(As):
        return v
    return P(Head(As), FoldRight(P, Tail(As), v))
\end{python}

Usando la funzione {\tt FoldRight}, si scrivano le funzioni seguenti:

\begin{enumerate}
\item {\tt And(Ls)}: prende in input una lista di valori booleani e ne calcola l'{\tt and} complessivo
\item {\tt Or(Ls)}: prende in input una lista di valori booleani e ne calcola l'{\tt or} complessivo
\item {\tt FoldFactorial(n)}: prende in input un numero intero $n$ e calcola il fattoriale $n!$
\end{enumerate}


%%%%%%%%%%%%%%%%%%%%%%%%%%%%%%%%%%%%%%%%%%%%%%%%%%%%%%%%%%%%%%%%%%%%%%%%%%%%%
\item Scrivere una funzione {\tt Str2Int(s)} che prende in input una stringa {\tt s} di sole cifre (numeri da 0 a 9) e converte la stringa in un numero intero in base 10. Ad esempio, la stringa "123" deve dare il numero $100+20+3 = 123$.


%%%%%%%%%%%%%%%%%%%%%%%%%%%%%%%%%%%%%%%%%%%%%%%%%%%%%%%%%%%%%%%%%%%%%%%%%%%%%
\item Scrivere una funzione {\tt Int2Str(n)} che prende in input un numero intero $n$ e lo converte in una stringa.


%%%%%%%%%%%%%%%%%%%%%%%%%%%%%%%%%%%%%%%%%%%%%%%%%%%%%%%%%%%%%%%%%%%%%%%%%%%%%
\item Scrivere un predicato {\tt IsUpper(s)} che prende in input una stringa e restituisce {\tt True}
se la stringa contiene solo caratteri alfabetici (da 'A' fino a 'Z', alfabeto inglese) maiuscoli.

%%%%%%%%%%%%%%%%%%%%%%%%%%%%%%%%%%%%%%%%%%%%%%%%%%%%%%%%%%%%%%%%%%%%%%%%%%%%%
\item Scrivere un predicato {\tt IsLower(s)} che prende in input una stringa e restituisce {\tt True}
se la stringa contiene solo caratteri alfabetici minuscoli (da 'a' fino a 'z', alfabeto inglese).
\end{enumerate}

\end{document}
