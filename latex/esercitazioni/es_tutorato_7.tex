\documentclass[11pt,a4]{article}

\usepackage[margin=2cm]{geometry}

%% Language and font encodings
\usepackage[english]{babel}
\usepackage[T1]{fontenc}
\usepackage{graphicx}
\usepackage{graphicx, subfigure}

\usepackage{multicol}

\usepackage{amsmath}
\usepackage{url}

\usepackage{hyperref}

\usepackage{amssymb}

\usepackage{amsmath}
\usepackage{url}

\usepackage[utf8]{inputenc}

% Default fixed font does not support bold face
\DeclareFixedFont{\ttb}{T1}{txtt}{bx}{n}{10} % for bold
\DeclareFixedFont{\ttm}{T1}{txtt}{m}{n}{10}  % for normal

% Custom colors
\usepackage{color}
\definecolor{deepblue}{rgb}{0,0,0.5}
\definecolor{deepred}{rgb}{0.6,0,0}
\definecolor{deepgreen}{rgb}{0,0.5,0}

\usepackage{listings}

% Python style for highlighting
\newcommand\pythonstyle{\lstset{
language=Python,
basicstyle=\ttm,
otherkeywords={self},             % Add keywords here
keywordstyle=\ttb\color{deepblue},
emph={MyClass,__init__},          % Custom highlighting
emphstyle=\ttb\color{deepred},    % Custom highlighting style
stringstyle=\color{deepgreen},
frame=tb,                         % Any extra options here
showstringspaces=false            % 
}}


% Python environment
\lstnewenvironment{python}[1][]
{
\pythonstyle
\lstset{#1}
}
{}

% Python for external files
\newcommand\pythonexternal[2][]{{
\pythonstyle
\lstinputlisting[#1]{#2}}}

% Python for inline
\newcommand\pythoninline[1]{{\pythonstyle\lstinline!#1!}}


\usepackage{collectbox}

\newcommand{\mybox}[2]{$\quad$\fbox{
\begin{minipage}{#1cm}
\hfill\vspace{#2cm}
\end{minipage}
}}


\usepackage{fancyhdr}
\pagestyle{fancy}
\rhead{Programmazione 1 - Tutorato 7}

\usepackage[T1]{fontenc}
\usepackage[utf8]{inputenc}
\usepackage{lmodern}
%%%%%%%%%%%%%%%%%%%%%%%%%%%%%%%%%%%%%%%%%%%%%%%%%%%%%%%%%
% Source: http://en.wikibooks.org/wiki/LaTeX/Hyperlinks %
%%%%%%%%%%%%%%%%%%%%%%%%%%%%%%%%%%%%%%%%%%%%%%%%%%%%%%%%%
\usepackage{hyperref}
\usepackage{graphicx}
\usepackage[english]{babel}

\usepackage{bm}
\usepackage{amsmath}
\usepackage{amsfonts}

\usepackage{amsthm}
\newtheorem{definition}{Definizione}
\newtheorem{theorem}{Teorema}
\renewcommand*{\proofname}{Dimostrazione}
\newtheorem{example}{Esempio}
\newtheorem{lemma}{Lemma}
\newtheorem{exercise}{Esercizio}
\newtheorem{property}{Proprietà}

\usepackage[ruled,vlined,linesnumbered]{algorithm2e}

\newcommand{\xstar}{x^*}
\newcommand{\bxstar}{\bm{x^*}}
\newcommand{\bx}{\bm{x}}
\newcommand{\Rn}{\mathbb{R}^n}
\newcommand{\RR}{\mathbb{R}}
\newcommand{\norm}[1]{\left\lvert \left\lvert #1 \right\lvert \right\lvert}

\newcommand{\fx}{f(x)}

\newcommand{\gradfx}{\nabla \fx}
\newcommand{\Gx}{\nabla f(x)}
\newcommand{\Gk}{\nabla f(x_k)}
\newcommand{\Gs}{\nabla f(\xstar)}

\newcommand{\Hx}{\nabla^2 f(x)}
\newcommand{\Hk}{\nabla^2 f(x_k)}
\newcommand{\Hs}{\nabla^2 f(\xstar)}
\newcommand{\hess}{\nabla^2 f}

\newcommand{\step}{\alpha}
\newcommand{\Seqx}{\{ x_k \}}

\usepackage{mathtools}
\newcommand\myeq{\stackrel{\mathclap{\normalfont\mbox{def}}}{=}}

\usepackage{listings}
\lstset
{ 
    language=Matlab,
    basicstyle=\normalsize,
    numbers=left,
    stepnumber=1,
    showstringspaces=false,
    tabsize=1,
    breaklines=true,
    breakatwhitespace=false,
   frame=single
}


\begin{document}
\thispagestyle{empty}
\hrule
\begin{center}
   {\Large {\bf Programmazione 1 \hspace{3cm} $\quad \quad \quad$ Tutorato 7}}
\end{center}

\hrule

%%%%%%%%%%%%%%%%%%%%%%%%%%%%%%%%%%%%%%%%%%%%%%%%%%%%%%%%%%%%%%%%%%%%%%%%%%%%%
\section*{Gun violence Data - 2013/2018}



\begin{enumerate}

\item Leggere il File \url{gun-violence-data_01-2013_03-2018.csv} che trovate sul sito \url{https://www.kaggle.com/jameslko/gun-violence-data/data} e memorizzare l'essenziale (specificato negli esercizi successivi) in una struttura dati opportuna: ad esempio una lista di liste, o una lista di oggetti. Il DataSet contiene le
caratteristiche di più di 260 mila incidenti avvenuti con arma da fuoco negli stati uniti dal 2013 in poi. Consultare la documentazione fornita sul sito per ulteriori informazioni.\\

\textbf{Attenzione}: il Dataset potrebbe contenere errori.

%%%%%%%%%%%%%%%%%%%%%%%%%%%%%%%%%%%%%%%%%%%%%%%%%%%%%%%%%%%%%%%%%%%%%%%%%%%%

\item Scrivere una funzione CountIn(Rs) che prende in input la lista restituita dalla funzione
ParseIncident e conti quanti incidenti sono avvenuti nel 2014 in Alabama, e negli Stati Uniti.

\item Scrivere una funzione mdState(Rs) che restituisca una \textit{lista} o un \textit{dizionario} coi 5 stati in cui è avvenuto il maggior numero di incidenti nel 2014. Verificare il risultato:
\begin{center}
[('California', 65),
 ('Texas', 60),
 ('Florida', 59),
 ('Pennsylvania', 44),
 ('Alabama', 27)]
\end{center}

\item L'intento ora è capire -se c'è- in quale periodo dell'anno è concentrato il maggior numero di omicidi. Per farlo restringiamo l'attenzione ad un anno particolare, ad esempio il 2014: ponendo sull'asse delle ascisse i giorni dell'anno plottare il numero di omicidi in quel giorno calcolato su tutto il terrirorio degli Stati Uniti. Verificare un andamento del tipo:

\begin{figure}[!ht]
\centering
\subfigure[\textbf{Omicidi nel 2014} ]{
\includegraphics[width=0.6\textwidth]{plot_2013.png}}
\end{figure}


\end{enumerate}










\end{document}