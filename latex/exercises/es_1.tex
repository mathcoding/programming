\documentclass[11pt,a4]{article}

\usepackage[margin=2cm]{geometry}


\usepackage{collectbox}

\newcommand{\mybox}[2]{$\quad$\fbox{
\begin{minipage}{#1cm}
\hfill\vspace{#2cm}
\end{minipage}
}}

\usepackage{fancyhdr}
\pagestyle{fancy}
\rhead{Programmazione 1 - Esercitazione 1, A.A. 2018-2019}

\usepackage[T1]{fontenc}
\usepackage[utf8]{inputenc}
\usepackage{lmodern}
%%%%%%%%%%%%%%%%%%%%%%%%%%%%%%%%%%%%%%%%%%%%%%%%%%%%%%%%%
% Source: http://en.wikibooks.org/wiki/LaTeX/Hyperlinks %
%%%%%%%%%%%%%%%%%%%%%%%%%%%%%%%%%%%%%%%%%%%%%%%%%%%%%%%%%
\usepackage{hyperref}
\usepackage{graphicx}
\usepackage[english]{babel}

\usepackage{bm}
\usepackage{amsmath}
\usepackage{amsfonts}

\usepackage{amsthm}
\newtheorem{definition}{Definizione}
\newtheorem{theorem}{Teorema}
\renewcommand*{\proofname}{Dimostrazione}
\newtheorem{example}{Esempio}
\newtheorem{lemma}{Lemma}
\newtheorem{exercise}{Esercizio}
\newtheorem{property}{Proprietà}

\usepackage[ruled,vlined,linesnumbered]{algorithm2e}

\newcommand{\xstar}{x^*}
\newcommand{\bxstar}{\bm{x^*}}
\newcommand{\bx}{\bm{x}}
\newcommand{\Rn}{\mathbb{R}^n}
\newcommand{\RR}{\mathbb{R}}
\newcommand{\norm}[1]{\left\lvert \left\lvert #1 \right\lvert \right\lvert}

\newcommand{\fx}{f(x)}

\newcommand{\gradfx}{\nabla \fx}
\newcommand{\Gx}{\nabla f(x)}
\newcommand{\Gk}{\nabla f(x_k)}
\newcommand{\Gs}{\nabla f(\xstar)}

\newcommand{\Hx}{\nabla^2 f(x)}
\newcommand{\Hk}{\nabla^2 f(x_k)}
\newcommand{\Hs}{\nabla^2 f(\xstar)}
\newcommand{\hess}{\nabla^2 f}

\newcommand{\step}{\alpha}
\newcommand{\Seqx}{\{ x_k \}}

\usepackage{mathtools}
\newcommand\myeq{\stackrel{\mathclap{\normalfont\mbox{def}}}{=}}

\usepackage{listings}
\lstset
{ 
    language=Matlab,
    basicstyle=\normalsize,
    numbers=left,
    stepnumber=1,
    showstringspaces=false,
    tabsize=1,
    breaklines=true,
    breakatwhitespace=false,
   frame=single
}


\begin{document}
\thispagestyle{empty}
\hrule
\begin{center}
   {\Large {\bf Esercitazione 1 \hspace{3cm} $\quad \quad$ Programmazione 1, a.a. 2018-2019}}
\end{center}
{\bf Cognome: }\hspace{2.5cm} {\bf Nome: } \hspace{2.5cm} {\bf Matricola: } \\\
\hrule

\begin{enumerate}
\section*{}

%%%%%%%%%%%%%%%%%%%%%%%%%%%%%%%%%%%%%%%%%%%%%%%%%%%%%%%%%%%%%%%%%%%%%%%%%%%%%
\item Scrivere una procedura ricorsiva chiamata {\tt Tab(x)} che prende in input un numero naturale {\tt x} e stampa a video
la tabellina del numero corrispondente sino a 10. Per esempio, applicando il valore 7 alla procedura {\tt Tab(x)}, 
la procedura stampa a video (un numero per riga):
\begin{verbatim}
7 14 21 28 .. 70
\end{verbatim}

\mybox{15}{3.5}

 
%%%%%%%%%%%%%%%%%%%%%%%%%%%%%%%%%%%%%%%%%%%%%%%%%%%%%%%%%%%%%%%%%%%%%%%%%%%%%
\item Il metodo di Newton per trovare la radice cubica di un numero si basa sul fatto che se
$y$ è un'approssimazione della radice cubica di x, allora un'approssimazione migliore è data
da $\frac{\frac{x}{y^2} + 2y}{3}$.
Si usi questa formula per implementare una procedura analoga a quella scritta per trovare
la radice quadrata. Si prenda spunto dalle soluzioni elaborate nel notebook Lab3.

\mybox{15}{4.5}


%%%%%%%%%%%%%%%%%%%%%%%%%%%%%%%%%%%%%%%%%%%%%%%%%%%%%%%%%%%%%%%%%%%%%%%%%%%%%
\item Si consideri l'algoritmo del contadino russo per moltiplicare due numeri interi positivi spiegato a lezione.
Si scriva una funzione chiamata {\tt MultiRec(a, b)} che prende in input due numeri interi {\tt a} e {\tt b}
e calcola il prodotto tra i due numeri usando l'algoritmo appena citato.
La funzione implementata deve avere un {\underline {\bf processo di calcolo ricorsivo lineare}} (si veda il notebook Lab4).

{\bf NOTA}: In Python per effettuare la divisione intera tra due numeri si usa l'operatore {\tt //} (e.g. {\tt 3 // 2 = 1}).
Per calcolare il resto di una divisione si usa l'operatore modulo {\tt \%} (e.g. {\tt 7\%4 = 3}). 
Si osservi che un numero {\tt x} è pari solo se {\tt x \% 2 == 0}.

\mybox{15}{3.5}

%%%%%%%%%%%%%%%%%%%%%%%%%%%%%%%%%%%%%%%%%%%%%%%%%%%%%%%%%%%%%%%%%%%%%%%%%%%%%
\item Si scriva una funzione {\tt MultiIter(a, b)} che implementa lo stesso algoritmo dell'esercizio precedente, ma con una procedura che realizza
un {\underline {\bf processo di calcolo iterativo lineare}} (si veda il notebook Lab4).

\mybox{15}{3.5}


%%%%%%%%%%%%%%%%%%%%%%%%%%%%%%%%%%%%%%%%%%%%%%%%%%%%%%%%%%%%%%%%%%%%%%%%%%%%%
\item Si scriva una funzione ricorsiva {\tt FibonacciIter(n)} che calcolo l'$n$-esimo numero della sequenza di Fibonacci
che realizza un {\underline {\bf processo di calcolo iterativo lineare}}.

\mybox{15}{3.5}

%%%%%%%%%%%%%%%%%%%%%%%%%%%%%%%%%%%%%%%%%%%%%%%%%%%%%%%%%%%%%%%%%%%%%%%%%%%%% 
\item Il Massimo Comun Divisore (MCD) di due numeri intero $a$ e $b$ è definito come 
il più grande numero intero che divide sia $a$ che $b$ senza resto. 
Esiste un metodo famoso, dovuto ad Euclide, per calcolare il MCD.
L'idea dell'algoritmo si basa sull'osservazione che, se $r$ è il resto di quando $a$ è diviso per $b$, 
allora i divisori comuni di $a$ e $b$ sono esattamente gli stessi divisori comuni tra $b$ e $r$. 
Quindi possiamo usare l'equazione $$MCD(a,b) = MCD(b,r)$$ per ridurre il problema di 
trovare i divisori comuni calcolando il MCD tra coppie di numeri interi via via più piccoli. Per esempio:

\begin{verbatim}
MCD(206, 40) = MCD(40, 6) = MCD(6,4) = MCD(4,2) = MCD(2,0) = 2
\end{verbatim}

Scrivere una procedura che calcola il massimo comune divisore usando l'algoritmo di Euclide.
{\bf NOTA}: per calcolare il resto di una divisione tra due numeri interi si usa l'operatore modulo {\tt \%}, 
ovvero il simbolo percentuale (esempio: {\tt 7\%3 = 1}).

\mybox{15}{3.5}

%%%%%%%%%%%%%%%%%%%%%%%%%%%%%%%%%%%%%%%%%%%%%%%%%%%%%%%%%%%%%%%%%%%%%%%%%%%%%
\item {\bf CHALLENGE (facoltativo)}: Si consideri il problema seguente. Siano date le monetine
da 1, 2, 5 e 10 centesimi di euro: quanti modi esistono per cambiare una monetina da 20 centesimi?
E se consideriamo anche le monetine da 20 e 50 centesimi, 
in quanti modi possiamo cambiare una moneta da un euro? 
Si utilizzino solo gli elementi del linguaggio Python visti a lezione. Mandare la soluzione per email al docente.

\end{enumerate}

\end{document}
