\documentclass[11pt,a4]{article}

\usepackage[margin=2cm]{geometry}

\usepackage[utf8]{inputenc}

% Default fixed font does not support bold face
\DeclareFixedFont{\ttb}{T1}{txtt}{bx}{n}{10} % for bold
\DeclareFixedFont{\ttm}{T1}{txtt}{m}{n}{10}  % for normal

% Custom colors
\usepackage{color}
\definecolor{deepblue}{rgb}{0,0,0.5}
\definecolor{deepred}{rgb}{0.6,0,0}
\definecolor{deepgreen}{rgb}{0,0.5,0}

\usepackage{listings}

% Python style for highlighting
\newcommand\pythonstyle{\lstset{
language=Python,
basicstyle=\ttm,
otherkeywords={self},             % Add keywords here
keywordstyle=\ttb\color{deepblue},
emph={MyClass,__init__},          % Custom highlighting
emphstyle=\ttb\color{deepred},    % Custom highlighting style
stringstyle=\color{deepgreen},
frame=tb,                         % Any extra options here
showstringspaces=false            % 
}}


% Python environment
\lstnewenvironment{python}[1][]
{
\pythonstyle
\lstset{#1}
}
{}

% Python for external files
\newcommand\pythonexternal[2][]{{
\pythonstyle
\lstinputlisting[#1]{#2}}}

% Python for inline
\newcommand\pythoninline[1]{{\pythonstyle\lstinline!#1!}}


\usepackage{collectbox}

\newcommand{\mybox}[2]{$\quad$\fbox{
\begin{minipage}{#1cm}
\hfill\vspace{#2cm}
\end{minipage}
}}

\usepackage{fancyhdr}
\pagestyle{fancy}
\rhead{Programmazione 1 - Esercitazione 1}

\usepackage[T1]{fontenc}
\usepackage[utf8]{inputenc}
\usepackage{lmodern}
%%%%%%%%%%%%%%%%%%%%%%%%%%%%%%%%%%%%%%%%%%%%%%%%%%%%%%%%%
% Source: http://en.wikibooks.org/wiki/LaTeX/Hyperlinks %
%%%%%%%%%%%%%%%%%%%%%%%%%%%%%%%%%%%%%%%%%%%%%%%%%%%%%%%%%
\usepackage{hyperref}
\usepackage{graphicx}
\usepackage[english]{babel}

\usepackage{bm}
\usepackage{amsmath}
\usepackage{amsfonts}

\usepackage{amsthm}
\newtheorem{definition}{Definizione}
\newtheorem{theorem}{Teorema}
\renewcommand*{\proofname}{Dimostrazione}
\newtheorem{example}{Esempio}
\newtheorem{lemma}{Lemma}
\newtheorem{exercise}{Esercizio}
\newtheorem{property}{Proprietà}

\usepackage[ruled,vlined,linesnumbered]{algorithm2e}

\newcommand{\xstar}{x^*}
\newcommand{\bxstar}{\bm{x^*}}
\newcommand{\bx}{\bm{x}}
\newcommand{\Rn}{\mathbb{R}^n}
\newcommand{\RR}{\mathbb{R}}
\newcommand{\norm}[1]{\left\lvert \left\lvert #1 \right\lvert \right\lvert}

\newcommand{\fx}{f(x)}

\newcommand{\gradfx}{\nabla \fx}
\newcommand{\Gx}{\nabla f(x)}
\newcommand{\Gk}{\nabla f(x_k)}
\newcommand{\Gs}{\nabla f(\xstar)}

\newcommand{\Hx}{\nabla^2 f(x)}
\newcommand{\Hk}{\nabla^2 f(x_k)}
\newcommand{\Hs}{\nabla^2 f(\xstar)}
\newcommand{\hess}{\nabla^2 f}

\newcommand{\step}{\alpha}
\newcommand{\Seqx}{\{ x_k \}}

\usepackage{mathtools}
\newcommand\myeq{\stackrel{\mathclap{\normalfont\mbox{def}}}{=}}

\usepackage{listings}
\lstset
{ 
    language=Matlab,
    basicstyle=\normalsize,
    numbers=left,
    stepnumber=1,
    showstringspaces=false,
    tabsize=1,
    breaklines=true,
    breakatwhitespace=false,
   frame=single
}


\begin{document}
\thispagestyle{empty}
\hrule
\begin{center}
   {\Large {\bf Programmazione 1 \hspace{3cm} $\quad \quad \quad$ Esercitazione 1}}
\end{center}
{\bf Cognome: }\hspace{2.5cm} {\bf Nome: } \hspace{2.5cm} {\bf Matricola: } \\\
\hrule

\begin{enumerate}
\section*{}

%%%%%%%%%%%%%%%%%%%%%%%%%%%%%%%%%%%%%%%%%%%%%%%%%%%%%%%%%%%%%%%%%%%%%%%%%%%%%
\item Scrivere una procedura che prenda tre numeri come argomenti 
e restituisca la somma dei quadrati dei due numeri più grandi.

\begin{python}
def SommaMinimi(a, b, c):
    if a < b and a < c:
        return b*b + c*c
    if b < a and b < c:
        return a*a + c*c
    return a*a + b*b
\end{python}

%%%%%%%%%%%%%%%%%%%%%%%%%%%%%%%%%%%%%%%%%%%%%%%%%%%%%%%%%%%%%%%%%%%%%%%%%%%%%
\item Si considerino le due procedure seguenti:

\begin{python}
def Inc(x):
    return x+1
def Dec(x):
    return x-1
\end{python}

Si scriva una funzione {\tt Somma(x,y)} che restituisca la somma {\tt x+y},
usando solo chiamate ricorsive, le espressioni condizionali, 
e le due funzioni {\tt Inc(x)} e {\tt Dec(x)}.

\begin{python}
def SommaRec(x, y):
    if y == 0:
        return x
    else:
        return SommaRec(Inc(x), Dec(y))
def Somma(x, y):
    if x < 0 or y < 0:
        print("Entrambi i numeri devono essere positivi!")
    else:
        return SommaRec(x,y)
\end{python}

%%%%%%%%%%%%%%%%%%%%%%%%%%%%%%%%%%%%%%%%%%%%%%%%%%%%%%%%%%%%%%%%%%%%%%%%%%%%%
\item Scrivere una o più procedure per trovare la radice quadrata di un numero, 
utilizzando il metodo di Newton descritto nel notebook {\tt Lab 2.ipynb}.
\begin{python}
def UgualeTol(x, y, tol=0.001):  # Uguali a meno di una tolleranza numerica
    return abs(x - y) < tol
def NewtonMigliora(y, x):
    return (y+x/y)/2
def NewtonIter(y, x):
    if UgualeTol(x, y*y):
        return y
    else:
        return NewtonIter(NewtonMigliora(y, x), x)
def RadiceNewton(x):
    return NewtonIter(1, x)
\end{python}
 
%%%%%%%%%%%%%%%%%%%%%%%%%%%%%%%%%%%%%%%%%%%%%%%%%%%%%%%%%%%%%%%%%%%%%%%%%%%%%
\item Il metodo di Newton per trovare la radice cubica di un numero si basa sul fatto che se
$y$ è un'approssimazione della radice cubica di x, allora un'approssi\-mazione migliore è data
da $\frac{\frac{x}{y^2} + 2y}{3}$.
Si usi questa formula per implementare una procedura analoga a quella scritta per trovare
la radice quadrata.

\begin{python}
def NewtonCuboMigliora(y, x):
    return (x/(y*y) + 2*y)/3
def NewtonCuboIter(y, x):
    if UgualeTol(x, y*y*y):
        return y
    else:
        return NewtonCuboIter(NewtonCuboMigliora(y, x), x)    
def RadiceCubicaNewton(x):
    return NewtonCuboIter(1, x)
\end{python}

%%%%%%%%%%%%%%%%%%%%%%%%%%%%%%%%%%%%%%%%%%%%%%%%%%%%%%%%%%%%%%%%%%%%%%%%%%%%%
\item Modificare le procedure {\tt RadiceBruteForce(x,y)}, {\tt RadiceBisection(x)} e la procedura
scritta per risolvere l'esercizio 1.2, in modo che vengano contate il numero di chiamate ricorsive.
Per ciascuno dei tre metodi, riportare nel riquadro sotto il numero di iterazioni necessarie
per trovare la radice quadrata di 13.

\begin{python}    
def CountRadiceBisection(x):
	def CountRadiceBisectionSearch(x, a, b, counter):
		if a > b:
			return -1
		y = (b + a)/2
		if UgualeTol(x, y*y):
			return y, counter
		if y*y < x:
			return CountRadiceBisectionSearch(x, y, b, counter+1)
		else:
			return CountRadiceBisectionSearch(x, a, y, counter+1)
    return CountRadiceBisectionSearch(x, 1, x, 1)

def CountRadiceNewton(x):
	def CountNewtonIter(y, x, counter):
		if UgualeTol(x, y*y):
			return y, counter
		else:
			return CountNewtonIter(NewtonMigliora(y, x), x, counter+1)    

	return CountNewtonIter(1, x, 1)
\end{python}

%%%%%%%%%%%%%%%%%%%%%%%%%%%%%%%%%%%%%%%%%%%%%%%%%%%%%%%%%%%%%%%%%%%%%%%%%%%%% 
\item Il Massimo Comun Divisore (MCD) di due numeri intero $a$ e $b$ è definito come 
il più grande numero intero che divide sia $a$ che $b$ senza resto. 
Esiste un metodo famoso, dovuto ad Euclide, per calcolare tale numero.
L'idea dell'algoritmo si basa sull'osservazione che, se $r$ è il resto di quando $a$ è diviso per $b$, 
allora i divisori comuni di $a$ e $b$ sono esattamente esattamente gli stessi divisori comuni tra $b$ e $r$. 
Quindi possiamo usare l'equazione $$MCD(a,b) = MCD(b,r)$$ per ridurre il problema di 
trovare i divisori comuni calcolando il MCD tra coppie di numeri interi via via più piccoli. Per esempio:

\begin{verbatim}
MCD(206, 40) = MCD(40, 6) = MCD(6,4) = MCD(4,2) = MCD(2,0) = 2
\end{verbatim}

Scrivere una procedura che calcola il massimo comune divisore usando l'algoritmo di Euclide.
{\bf NOTA}: per calcolare il resto di una divisione tra due numeri interi si usa l'operatore modulo {\tt \%}, 
ovvero il simbolo percentuale (esempio: {\tt 7\%3 = 1}).

\begin{python}
def MCD(a, b):
    if b == 0:
        return a
    else:
        return MCD(b, a % b)
\end{python}

%%%%%%%%%%%%%%%%%%%%%%%%%%%%%%%%%%%%%%%%%%%%%%%%%%%%%%%%%%%%%%%%%%%%%%%%%%%%%
\item {\bf CHALLENGE (facoltativo)}: Si consideri il problema seguente. Siano date le monetine
da 1, 2, 5 e 10 centesimi di euro: quanti modi esistono per cambiare una monetina da 20 centesimi?
E se consideriamo anche le monetine da 20 e 50 centesimi, 
in quanti modi possiamo cambiare una moneta da un euro? Mandare la soluzione per email.

\end{enumerate}

\end{document}
